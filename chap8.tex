\chapter{جمع‌بندی} \label{ch:conclusion}
در این بخش ابتدا به جمع‌بندی کارهای انجام شده می‌پردازیم. سپس، پیشنهادهایی برای تکمیل پروژه ارائه می‌کنیم.

\section{کارهای انجام شده}
در این پژوهش روشی برای تقطیع تصاویر مبتنی بر کاهش افزونگی پیشنهاد شد.
بردارهای ویژگی استخراج شده از بافت‌های تصویر با توزیع آمیخته
\lr{von Mises-Fisher}
مدل شد.
در الگوریتم پیشنهادی، ابتدا تصویر با یک الگوریتم تقطیع سطح پایین به تعداد زیادی ابرپیکسل تقطیع شده و با برازش مدل به بردارهای ویژگی و تعریف طول کد لازم برای بازنمایی قطعات با معیار کاهش افزونگی مبتنی بر درست‌نمایی مدل، قطعات مجاور، با هدف کاهش طول کد کل با یکدیگر ادغام می‌شوند.



یک بسته نرم‌افزاری جهت تسهیل کار با مدل‌های آمیخته در ضمن کار روی پروژه تهیه شد که امکانات بسیاری برای پژوهشگران در برازش این مدل‌ها به داده‌های مورد پژوهش فراهم می‌کند.
ویژگی اختصاصی این جعبه ابزار، امکان استفاده از
بهینه‌سازی ریمانی می‌باشد که قابلیت اجرای الگوریتم‌های مبتنی بر گرادیان و انواع تکنیک‌های مرتبط با آن‌ها را برای کاربر فراهم می‌آورد.
این قابلیت می‌تواند راه را برای استفاده از مدل‌های آمیخته در مسائل با مقیاس بزرگ هموار سازد.

همچنین الگوریتم
\lr{k-means++}~\cite{arthur_kmeans_2007}
برای مقداردهی اولیه پارامترهای توزیع آمیخته vMF در بهینه‌سازی، تعمیم داده شد و حد بالایی برای تابع هزینه بهینه‌سازی مورد اثبات قرار گرفت.



\section{کارهای آینده}
\iffalse
با توجه به تاثیر پارامتر
$\lambda$
بر اندازه قطعات ایجاد شده، احتمالا طراحی یک الگوریتم برای تنظیم انطباق‌پذیر این پارامتر به گونه‌ای که در نواحی دارای جزئیات بالا در تصویر مقدار آن کاهش یافته و در نواحی گسترده و با جزئیات کمتر، مقدار پارامتر افزایش یابد می‌تواند در کیفیت بهتر قطعات ایجاد شده تاثیرگذار باشد.
\fi

در این پژوهش ما فقط مسئله تقطیع بدون نظارت تصاویر طبیعی را مورد مطالعه قرار دادیم.
اما چارچوب پیشنهاد شده قابلیت تعمیم به یادگیری با نظارت را نیز داراست.
به نظر ما نگاه به مسئله تقطیع تصاویر از منظر کاهش افزونگی می‌تواند در بهبود درک ما از تقطیع طبیعی تصاویر توسط موجودات زنده مفید باشد.
این ادراک می‌تواند بینش خوبی برای حل مسائل مهم بینایی ماشین مانند
{\mmm{کشف اشیای برجسته}{salient object detection}}،
{\mmm{مرتب‌سازی ادراکی}{perceptual organization}} و
{\mmm{ادراک و حاشیه‌نویسی تصاویر}{image understanding and annotation}}
برای پژوهشگران این حوزه داشته باشد.

در خصوص بسته نرم‌افزاری تهیه‌شده، فضای فراوانی برای توسعه وجود دارد.
به عنوان مثال می‌توان به پیاده‌سازی انواع توزیع‌ها و الگوریتم‌های تخمین پارامتر،
پشتیبانی از جزئیات کاربردهای گوناگون مدل‌های آمیخته در حوزه‌های مختلف
و نیز گسترش محیط‌های نرم‌افزاری پشتیبانی‌شده توسط این مجموعه اشاره نمود.




