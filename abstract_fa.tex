\thispagestyle{empty}
\baselineskip=1.5cm
%\chapter*{\centerline{چکیده}}
\parbox{\linewidth}{\centering \textbf{\Large{چکیده}}}
\noindent
در این پژوهش، ما مسئله تقطیع تصاویر را به صورت یک مسئله خوشه‌بندی بردارهای ویژگی استخراج شده از بافت‌های تصویر بازتعریف می‌کنیم.
فرض می‌کنیم که این داده‌ها، دارای توزیع آمیخته‌ای از توزیع‌های
\lr{von Mises-Fisher}
هستند.
در الگوریتم پیشنهادی ما به وسیله برازش این مدل‌ها به داده‌ها و با استفاده از یک روش خوشه‌بندی ادغامی مبتنی بر درست‌نمایی برازش، اقدام به تقطیع تصویر می‌نماییم.
کارایی الگوریتم بر حسب شاخص‌های کمی در کنار ارزیابی بصری، با انجام آزمایش‌های گسترده‌ای مورد بررسی و مقایسه با برخی از سایر الگوریتم‌های تقطیع تصویر قرار می‌گیرد.
یک بسته نرم‌افزاری به منظور کار با مدل‌های آمیخته در ضمن کار روی این پروژه تهیه شده است که قابلیت‌های مورد نیاز جهت تخمین پارامترهای این مدل‌ها را فراهم می‌کند.

\vspace{1cm}
\noindent
\textbf{کلمات کلیدی:} \textit{
تقطیع تصاویر؛ تقطیع بافت؛ مدل آمیخته؛ توزیع
\lr{von Mises-Fisher}؛
خوشه‌بندی؛ جعبه ابزار
\lr{MATLAB}؛
بهینه‌سازی روی منیفلد
}

\baselineskip=1cm
