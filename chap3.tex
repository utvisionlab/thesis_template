\chapter{تقطیع تصاویر} \label{ch:imageSegmentation}
%from Zhang book
تکنیک‌های
{\mmm{مهندسی تصاویر}{image engineering}}
را می‌توان به سه لایه کلی تقسیم نمود:
{\mmm{پردازش تصاویر}{image processing}}،
{\mmm{تحلیل تصاویر}{image analysis}}
و
{\mmm{ادراک تصاویر}{image understanding}}%
~\cite{zhang_advances_2006}.
تقطیع تصاویر، اولین گام لازم برای تحلیل تصاویر است و انجام درست آن تاثیر بسزایی در دقت تحلیل خواهد داشت. در تحلیل تصاویر، هدف، استخراج اطلاعات مورد نیاز از تصاویر است که در سه مرحله صورت می‌گیرد: تقطیع تصویر،
{\mmm{بازنمایی اشیا}{object representation}}
و
{\mmm{سنجش مشخصات}{feature measurement}}.

%TODO (from Szeliski p237)
\iffalse
کار مشابه تقطیع تصاویر در علم آمار،
{\mmm{آنالیز خوشه‌ها}{cluster analysis}}
است که پژوهش‌های زیادی روی آن انجام شده و صدها الگوریتم برای آن پیشنهاد شده است.
\fi

تقطیع تصاویر معمولا به این صورت تعریف می‌شود: فرایند تقسیم‌بندی تصویر به بخش‌های تشکیل‌دهنده آن و استخراج بخش‌های مورد نظر (اشیاء).
نتایج تقطیع در کلیه مراحل بعدی تحلیل تصویر مانند بازنمایی، توصیف و سنجش مشخصات اشیا، و حتی فرایندهای سطوح بالاتر مانند طبقه‌بندی اشیا و تفسیر صحنه تاثیرگذار خواهد بود.
اولین تکنیک برای تجزیه تصاویر به بخش‌های تشکیل‌دهنده آن‌ها توسط
\lr{Roberts}
معرفی شد%
~\cite{zhang_advances_2006}.
او یک اپراتور برای آشکارسازی لبه‌های میان بخش‌های مختلف تصویر معرفی کرد که با نام خودش شناخته می‌شود\LTRfootnote{Roberts edge detector}.
پس از آن، تکنیک‌ها و الگوریتم‌های زیادی برای تقطیع تصاویر پیشنهاد شده است.
%from zhu_pixels_2015
در روش‌های اولیه، با الهام از ایده‌های موجود برای خوشه‌بندی، تاکید بر
{\mmm{ادغام}{merge}}
و
{\mmm{انشعاب}{split}}
نواحی به صورت محلی بود%
~\cite{ohlander_picture_1978, brice_scene_1970}.
اما در روش‌های جدید معمولا یک معیار سراسری در کل تصویر بهینه‌سازی می‌شود%
~\cite{comaniciu_mean_2002, felzenszwalb_efficient_2004, shi_normalized_2000, chan_active_2001, beare_locally_2006}.
همچنین روش‌های تقطیع
{\mmm{تعاملی}{interactive}}
پیشنهاد شده‌اند که در کاربردهایی مانند تحلیل تصاویر پزشکی یا ویرایش تصاویر که در آن‌ها مکان‌یابی دقیق اشیاء در تصویر مورد نیاز است، با کمک گرفتن از ورودی کاربر عمل تقطیع را انجام می‌دهند%
~\cite{rother_grabcut_2004, nguyen_robust_2012}.
با پیدایش مجموعه دادگان مقیاس بزرگ برای تصاویر، مانند مجموعه
ImageNet~\cite{deng_imagenet_2009}،
و نیز تصاویر روزافزون به اشتراک گذاشته شده در شبکه‌های اجتماعی اینترنتی، الگوریتم‌های
{\mmm{تقطیع مشترک}{cosegmentation}}
که اشیاء تکرار شده را در مجموعه‌ای از تصاویر جدا می‌کنند، در سال‌های اخیر مورد توجه زیادی قرار گرفته‌اند%
~\cite{rother_cosegmentation_2006, kim_multiple_2012, joulin_multiclass_2012}.

در این مدت همچنین مفهوم و دامنه تصاویر نیز گسترش زیادی یافته است.
به عنوان مثال تصاویر دوبعدی به تصاویر سه‌بعدی و تصاویر ثابت به تصاویر متحرک (ویدئو) و تصاویر
{\mmm{طیف خاکستری}{gray level}}
به تصاویر رنگی و
{\mmm{چندطیفی}{multi-band}}
توسعه پیدا کرده‌اند
که این موضوع به نوبه خود به گسترش مفاهیم و تکنیک‌های تقطیع تصاویر منجر شده است.

می‌توان تقطیع را به صورت افراز تصویر به تعدادی ناحیه ناهمپوشان که اجتماع آنها کل تصویر را می‌سازد در نظر گرفت.
\lr{Haralick}
قوانین زیر را برای نواحی حاصل از تقطیع تصاویر پیشنهاد کرده است%
~\cite{haralick_image_1985}:

\begin{enumerate}
	\item
	هر ناحیه باید از نظر ویژگی‌های مشخصی یکنواخت و همگن باشد.
	\item
	نباید در داخل نواحی، حفره‌های زیادی وجود داشته باشد.
	\item
	تفاوت نواحی مجاور از نظر ویژگی‌هایی که برای هر کدام یکنواخت است، باید به طور معناداری زیاد باشد.
	\item
	مرزهای هر ناحیه باید ساده و غیر مجعد بوده و به طور دقیق مشخص شده باشند.
\end{enumerate}

% Three Levels of Research (p19)
کارهای انجام شده در مورد تقطیع تصاویر، عموما طراحی شگردهایی برای کاربردهای خاص بوده و تا کنون نیز نظریه‌ای عمومی در این خصوص ارائه نشده است.
تحقیقات در این خصوص با جهت‌گیری‌ها و مبانی گوناگون  انجام شده و در نتیجه، الگوریتم‌های تقطیع بسیار متنوعی در ادبیات این حوزه پیشنهاد شده است
که نه هیچ یک را می‌توان به طور عمومی روی کلیه تصاویر اعمال کرد و نه می‌توان گفت که الگوریتم‌های متفاوت، همه برای یک کاربرد خاص مناسب هستند.
در ادامه این فصل به مرور برخی از شناخته‌شده‌ترین روش‌های تقطیع خواهیم پرداخت.


\section{مرور برخی از روش‌های رایج تقطیع}
پژوهش‌های زیادی در دهه‌های اخیر در زمینه تقطیع تصاویر طبیعی انجام شده است.
برخی از آن‌ها توانسته‌اند موفقیت و محبوبیت زیادی کسب کرده و در کاربردهای وسیعی مورد استفاده قرار بگیرند.

ویژگی‌های زیر از خصوصیات مهم شاخص NPR به شمار می‌روند. سه ویژگی اول با شاخص PR مشترک هستند اما ویژگی چهارم مختص معیار NPR می‌باشد.

\begin{description}
	\item[عدم امکان تباهیدگی:]
	هنگامی که تقطیع ورودی قابلیت بازنمایی مناسب با مجموعه تقطیع پایه را نداشته باشد، باز هم مقادیر شباهت در محدوده‌ای معقول خواهند بود.
	\item[عدم نیاز به پیش‌فرض‌های خاص در مورد داده‌ها:]
	برای استفاده از این معیار نیازی نیست که تعداد نواحی یا اندازه آن‌ها در دو تقطیع با هم برابر باشند.
	\item[انطباق‌پذیری پویا با جزئیات:]
	این معیار، در ناحیه‌هایی که در تقطیع دستی مبهم بوده‌اند، تفاوت‌های کوچک در سطح پیکسل را در نظر نگرفته و در نواحی دیگر آن‌ها را جریمه می‌کند.
	\item[نمرات قابل مقایسه:]
	نمراتی که با این معیار به دست می‌آیند، امکان مقایسه بامعنا بین تقطیع‌های تصاویر مختلف و نیز تقطیع‌های مختلف یک تصویر واحد را فراهم می‌کنند.
\end{description}

ارزش و کارایی هر معیار تشابه، مبتنی بر مقدار مرجعی است که نسبت به آن سنجیده می‌شود.
برای مسئله‌ی تقطیع تصاویر، می‌توان این مقدار را برابر مقدار مورد انتظار معیار با استفاده از تغییرات و واریانس موجود در مجموعه 






